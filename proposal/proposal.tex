\documentclass[12pt]{article}
\usepackage{url}
\usepackage{amsmath,amsthm,enumitem}
\usepackage{geometry}
\geometry{left=1.5cm, right= 1.5cm, top=2.5cm, bottom=2.5cm}
\usepackage{graphicx}
\usepackage{float}
\usepackage{hyperref}
\usepackage{indentfirst}
\title{CS6630 Project Proposal\\
       Visualization for flights ( ) of United States}
\author{Run Li, Yulong Liang, Zhi Wang}

\begin{document}

\maketitle

\section{Basic Information}
    $\diamond$ The project title is "Visualization of flights () of United States".

    $\diamond$ Group members:

    $\quad\cdot$ Run Li, u0879939, $u0879939@utah.edu$

    $\quad\cdot$ Yulong Liang,

    $\quad\cdot$ Zhi Wang,

    $\diamond$ Project repository link: \url{https://github.com/zhiwang93/CS6630Project}

\section{Background and Motivation}
    Visualization can always give people more intuitional feelings about the data. the department of aviation in each country collects a great amount of data from their flights and airports. These data are necessary for the management of flights, the safety of passengers, and the improvement of the quality of aviation services. As an ordinary passenger, however, we may not realize how busy the sky is. We may often meet with flight delays, flight cancellations and other cases which we thought to be unexpected. But the visualization of data will tell us those unexpected cases are actually quite predictable. In this project, we will explore the data of all flights and airports in United States in the year of 2016. We expect to show some relationship between distributions of flights and their spatial and temporal conditions, so as to bring some intuitional feelings to passengers about how our aviation system is running everyday.
\section{Project Objectives}
    \noindent In this project, we expected to show:\\   
    $\diamond$ The connection of each airport to other airports.\\
    $\diamond$ The distribution of flights of each airport in each time period.\\
    $\diamond$ A comparison between planned departing time and actual departing time.\\
    $\diamond$ A comparison between planned arriving time and actual arriving time.\\
    $\diamond$ The rate of diversion and cancellation.\\
    $\diamond$ The expected delay.\\
    $\diamond$ A view of the time evolution of the flights.
    
    With these visualizations, we will show how complicated the aviation system is and to reveal some relationship between the distribution of flights and their spatial and temporal conditions. We will see which area has the highest density of flights, which airport is the busiest, at what time we may expect a delayed flight, etc.    
    
\section{Data}
    We collect our data from \url{www.transtats.bts.gov}.
\section{Data Processing}
\section{Visualization Design}
\section{Must-Have Features}
\section{Optional Features}
\section{Project Schedule}

\end{document}
